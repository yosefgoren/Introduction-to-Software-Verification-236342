\documentclass{article}
% basics
\usepackage{amsfonts}
\usepackage{enumitem}
\usepackage{float}
\usepackage{graphicx}
\usepackage{hyperref} 
\usepackage[labelfont=bf]{caption}

\newtheorem{theorem}{Theorem}
\newtheorem{lemma}[theorem]{Lemma}
\newtheorem{corollary}{Corollary}[theorem]

% unique math expressions:  
\usepackage{amsmath}
\DeclareMathOperator*{\andloop}{\wedge}
\DeclareMathOperator*{\pr}{Pr}
\DeclareMathOperator*{\approach}{\longrightarrow}
\DeclareMathOperator*{\eq}{=}

% grey paper
\usepackage{xcolor}
% \pagecolor[rgb]{0.11,0.11,0.11}
% \color{white}

% embedded code sections
\usepackage{listings}
\definecolor{codegreen}{rgb}{0,0.6,0}
\definecolor{codegray}{rgb}{0.5,0.5,0.5}
\definecolor{codepurple}{rgb}{0.58,0,0.82}
\lstdefinestyle{mystyle}{
    commentstyle=\color{codegreen},
    keywordstyle=\color{magenta},
    numberstyle=\tiny\color{codegray},
    stringstyle=\color{codepurple},
    basicstyle=\ttfamily\footnotesize,
    breakatwhitespace=false,         
    breaklines=true,                 
    captionpos=b,                    
    keepspaces=true,                 
    numbers=left,                    
    numbersep=5pt,                  
    showspaces=false,                
    showstringspaces=false,
    showtabs=false,                  
    tabsize=2
}

\lstset{style=mystyle}

\begin{document}
\author{Yosef Goren}
\title{Introduction to Software Verification 236342, Homework 1}
\maketitle
\section{}
%enumerate with capital letters:
\begin{enumerate}[label=\Alph*.]
    \item Correct. Since the precondition is false, the postcondition is 'always' satisfied (since it is never tested).
    \item Incorrect. Counterexample $x=-100,y=-99$:
    \begin{itemize}
        \item $l_0, -100, -99$
        \item $l_1, -100, -99$
        \item $l_2, -100, -99$
        \item $l_3, -1, -99$
        \item $l_*, -1, -99$
    \end{itemize}
    As can be seen, precondition is satisfied and postcondition is not.
    \item Correct. The postcondition is true, so regardless of anything else,
    for every input selection it will be evaluated as true (the program does not even have to terminate either).
    \item Incorrect. Counterexample $x=1, y=9$:
    \begin{itemize}
        \item $l_0, 1, 9$
        \item $l_1, 1, 9$
        \item $l_2, -8, 9$
        \item $l_3, -8, 9$
        \item $l_*, -8, 9$
    \end{itemize}
    Postcondition is false, so it is not satisfied.
    \item Incorrect. Counterexample $x=1, y=3$:
    \begin{itemize}
        \item $l_0, 1, 3$
        \item $l_1, 1, 3$
        \item $l_2, 1, 3$
        \item $l_3, -2, 3$
        \item $l_4, -2, 3$
        \item $l_1, -2, -3$
        \item $l_2, -2, -3$
        \item $l_3, -2, -3$
        \item $l_4, -2, -3$
        \item $l_1, -2, 3$
        \item $l_2, -2, 3$
        \item $l_3, -5, 3$
        \item $l_4, -5, 3$
    \end{itemize}
    \item Incorrect. Counterexample $x=1,y=2$,
    By running this example we see that the program gets stuck in a loop
    at labels $l_1, l_2, l_3, l_4$. And each 4 iterations result with the state
    being the same as the initial state at $l_1$.
    Since this is a total correcness condition on the specification,
    the correctness is contredicted by the program failing to terminate.
\end{enumerate}

\section{}

\section{}
\begin{enumerate}[label=\Alph*.]
    \item To enforce the program to not finish
    on a spesific set on inputs, we can require that if
    said inputs have been given and the program finishes - 
    the postcondition fails:
    \[\{\forall p\in\mathbf{P}, x=p^2\}P\{false\}\]
    \item To require that for a set of inputs a program finishes.
    we can use the precondition to apply the condition only to
    the relivant set and use total correctness to require the program to halt
    on these inputs:
    \[<gcd(x,y)=1>P<true>\]
\end{enumerate}

\end{document}