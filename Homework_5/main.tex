\documentclass{article}
% basics
\usepackage{amsfonts}
\usepackage{enumitem}
\usepackage{float}
\usepackage{graphicx}
\usepackage{hyperref} 
\usepackage[labelfont=bf]{caption}
\usepackage{algpseudocode}
\usepackage{verbatim}

\newtheorem{theorem}{Theorem}
\newtheorem{lemma}[theorem]{Lemma}
\newtheorem{corollary}{Corollary}[theorem]

% unique math expressions:  
\usepackage{amsmath}
\DeclareMathOperator*{\andloop}{\wedge}
\DeclareMathOperator*{\pr}{Pr}
\DeclareMathOperator*{\approach}{\longrightarrow}
\DeclareMathOperator*{\eq}{=}

% grey paper
\usepackage{xcolor}
% \pagecolor[rgb]{0.11,0.11,0.11}
% \color{white}

% embedded code sections
\usepackage{listings}
\definecolor{codegreen}{rgb}{0,0.6,0}
\definecolor{codegray}{rgb}{0.5,0.5,0.5}
\definecolor{codepurple}{rgb}{0.58,0,0.82}
\lstdefinestyle{mystyle}{
    commentstyle=\color{codegreen},
    keywordstyle=\color{magenta},
    numberstyle=\tiny\color{codegray},
    stringstyle=\color{codepurple},
    basicstyle=\ttfamily\footnotesize,
    breakatwhitespace=false,         
    breaklines=true,                 
    captionpos=b,                    
    keepspaces=true,                 
    numbers=left,                    
    numbersep=5pt,                  
    showspaces=false,                
    showstringspaces=false,
    showtabs=false,                  
    tabsize=2
}

\usepackage{tikz}
\usetikzlibrary{calc,arrows,positioning}

\newcommand{\lv}[0]{\overline{v}}
\newcommand{\lx}[0]{\overline{x}}

\lstset{style=mystyle}

\begin{document}
\tikzset{
    treenode/.style = {align=center},
    c/.style = {treenode,circle,draw=black,minimum width=1.5em,minimum height=1.5em,text centered,font=\footnotesize},
    r/.style = {treenode,rectangle,draw=black,minimum width=1.5em,minimum height=1.5em,text centered,font=\footnotesize},
    level 1/.style={sibling distance=60mm},
    level 2/.style={sibling distance=30mm},
    level 3/.style={sibling distance=15mm},
    level 4/.style={sibling distance=7mm}
}
\author{Yosef Goren, Andrew Elashkin}
\title{Software Verification Homework 5}
\maketitle
\part{Dry}
\section{}
\subsection{}
\[
    R_{healthy}(u,v)=R(u,v)\wedge \neg R_{ill}(u,v)    
\]
This means a healthy transition is a transition
and it is also not an ill transition.

\subsection{}
In this question we assume $n=2^N$ for some $N\in\mathbb{N}$.\\
First we define a formula
for checking if a path $v_1,...,v_k$
respects $R_{ill}$:
\[
    HP(v_1,...,v_k)=\bigwedge_{i\in[k]}\bigwedge_{j\in[k]\setminus\{i\}}
        \left((v_i=v_j\wedge v_{i+1}=v_{j+1})\rightarrow \neg R_{ill}(v_i,v_{i+1})\right)
\]

Now checking if a path $v_1,...,v_k$
is a possible, respects $R_{ill}$
and satisfies $Gp$:\\
\[
    F(v_1,...,v_k)
    =I(v_0)\wedge HP(v_1,...,v_k)\wedge \bigwedge_{i\in[k]}p(v_i)
\]

Given this formula the algorithm is:
\begin{algorithmic}
    \For{$k\in[n+1]:$}
        \State $\phi:=F(v_1,v_2,...,v_k)\wedge\left(\bigvee_{i\in[n-1]}\left(v_k=v_i\wedge\bigwedge_{j=i}^{k-1}R_{healthy}(v_j,v_{j+1})\right)\right)$
        \If{$SAT(\phi)$}
            \State \Return $true$
        \EndIf
    \EndFor
    \State \Return $false$
\end{algorithmic}
If a path is possible, exists
a subpath of it that is no longer than $n$
vertecies - hence the algorithm will find
said path.
The condition appended to $F$ verifies 
that a path is not only
possible, repects $R_{ill}$ and satisfies $Gp$ -
but it also ends with a loop of healthy transitions
meaning it can be extended into an infinite
path with the desired properties.

\section{}
The requirements from a path to satisfy these requirements are:
\begin{enumerate}
    \item The path starts at an initial state.
    \item The transitions within the path are legal.
    \item The path ends with a loop (thus can be extended to an infinite path with the same local properties).
    \item For each vertex there is a son vertex that satisfies $p$.
    \item For each vertex there is a son vertex that satisfies $q$.
\end{enumerate}
Denote the set of vertecies in the graph with $\{u_i\}_{i\in[n]}$.\\
Now we provide a formula for each requirement:
\begin{enumerate}
    \item $I(v_0)$
    \item $\bigwedge_{i\in[k-1]}R(v_i,v_{i+1})$
    \item $\bigvee_{i\in[n-1]}v_k=v_i$
    \item $\bigvee_{i\in[k-1]}(\bigvee_{j\in[n]}R(v_i,v_j)\wedge p(v_j))$
    \item $\bigvee_{i\in[k-1]}(\bigvee_{j\in[n]}R(v_i,v_j)\wedge q(v_j))$
\end{enumerate}

So the formula for the path is:
\[
    F(v_1,...,v_k)=
    I(v_0)
    \wedge\left(\bigwedge_{i\in[k-1]}R(v_i,v_{i+1})\right)
\]\[
    \wedge\left(\bigvee_{i\in[n-1]}(v_k=v_i)\right)
    \wedge\left(\bigvee_{i\in[k-1]}(\bigvee_{j\in[n]}R(v_i,v_j)\wedge p(v_j))\right)
    \wedge\left(\bigvee_{i\in[k-1]}(\bigvee_{j\in[n]}R(v_i,v_j)\wedge q(v_j))\right)
\]
\newpage

Now the algorithm is:
\begin{algorithmic}
    \For{$k\in[n+1]:$}
        \State $\phi:=F(v_1,v_2,...,v_k)$
        \If{$SAT(\phi)$}
            \State \Return $true$
        \EndIf
    \EndFor
    \State \Return $false$
\end{algorithmic}

\part{Wet}
\setcounter{section}{0}
\section{}
The system is meant to verify a
username and passowrd passed as command line arguments.
The first parameter is a username
and the second one is a password;
both the username and password must be integers.

\section{}
The changes to the code have been:
\begin{itemize}
    \item Changed username to hardcoded ID.
    \item Changed pasword to \texttt{nondet\_int()}.
    \item Asserted that the verification failed (\texttt{assert(!verification\_success)}).
\end{itemize}
Also, when running the program we have passed \texttt{--trace}
which requires it to provide us with a counterexample in the
cases where the assertion fails.
The assertion failing means that the verification succeded -
hence a counterexample would include an assignment into
the password variable which results with a successful verification
meaning we have a correct password.

\section{}
To produce this password we have added an \texttt{assume(password != 16)} (which is what we got before 
for the same username),
which caused the program to produce an additional counterexample: 42.\\
To prove that this works for all usernames,
we have set:
\begin{itemize}
    \item \texttt{username=nondet\_int();}
    \item \texttt{password = 42;}
    \item \texttt{assert(verification\_success);},
\end{itemize}
So we are essentially asing cbmc to prove that for whatever username,
with password 42 - verification is successful (and it did).

\section{}
\subsection{}
The run finishes with \texttt{predictable.c}
but goes into an infinite loop with \texttt{binsearch.c}.

\subsection{}
If we add \texttt{--unwind 1},
the verification is successful for \texttt{binsearch.c}.
\texttt{--undwind} sets a limit
for how deep the verification should go into loops by
converting any loops into a series of nested if statements.\\
\texttt{--no-unwinding-assertions} means
that if the depth threashold is passed - 
it is not viewed as a failiure (as opposed to assuming the
program should not iterate for more than the given number of times).\\
The verification success means that no illegal memory accesses are possible within parts of the program that aren't deeper in a loop
than 1 iteration. (and we know nothing about
any part of the program that is deeper than one iteration into a loop).

\subsection{}
With \texttt{--unwind 2} the verification fails for \texttt{binsearch.c}.\\
This means that the programm can access memory illegally.
Upon further analysis, this is due to char overflow in the index variable
which results in neagtive indexation to the array:
\begin{verbatim}
Violated property:
    file binsearch.c line 12 function binsearch
    array `a' lower bound in a[(signed long int)middle]
    (signed long int)middle >= 0l
\end{verbatim}

\end{document}