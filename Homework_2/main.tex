\documentclass{article}
% basics
\usepackage{amsfonts}
\usepackage{enumitem}
\usepackage{float}
\usepackage{graphicx}
\usepackage{hyperref} 
\usepackage[labelfont=bf]{caption}

\newtheorem{theorem}{Theorem}
\newtheorem{lemma}[theorem]{Lemma}
\newtheorem{corollary}{Corollary}[theorem]

% unique math expressions:  
\usepackage{amsmath}
\DeclareMathOperator*{\andloop}{\wedge}
\DeclareMathOperator*{\pr}{Pr}
\DeclareMathOperator*{\approach}{\longrightarrow}
\DeclareMathOperator*{\eq}{=}

% grey paper
\usepackage{xcolor}
% \pagecolor[rgb]{0.11,0.11,0.11}
% \color{white}

% embedded code sections
\usepackage{listings}
\definecolor{codegreen}{rgb}{0,0.6,0}
\definecolor{codegray}{rgb}{0.5,0.5,0.5}
\definecolor{codepurple}{rgb}{0.58,0,0.82}
\lstdefinestyle{mystyle}{
    commentstyle=\color{codegreen},
    keywordstyle=\color{magenta},
    numberstyle=\tiny\color{codegray},
    stringstyle=\color{codepurple},
    basicstyle=\ttfamily\footnotesize,
    breakatwhitespace=false,         
    breaklines=true,                 
    captionpos=b,                    
    keepspaces=true,                 
    numbers=left,                    
    numbersep=5pt,                  
    showspaces=false,                
    showstringspaces=false,
    showtabs=false,                  
    tabsize=2
}

\lstset{style=mystyle}

\begin{document}
\author{Yosef Goren}
\title{}
\maketitle

\section*{Question 1}

\section*{Question 2}
Let $P\in FPG'$ (extended Flowchart Programming Languge - with interrupts).\\
And let $CFG:FPL\longrightarrow [n]\times [n]^2$ be a function returning
the control flow graph any program $P\in FPG$ (unmodified).\\

Define $CFG':FPL'\longrightarrow [n]\times [n]^2$ on input $(P,P_{int})$ to be the graph
$CFG(P)$ with the following modifications:
Any node $v\in P$ now has an additional output
goes into the initial
state of a private copy of $CFG(P_{int})$
(each $v\in P$ has it's own copy); Denote this private
copy with $S_v$. Additionally, all final states of $S_v$ go 
into a new node $F_v$, which goes back into $v$.\\

Define a reachability condition $R$ and state transformation $T$ for any path $\tau$ of $FPL'$ program graphs as follows:
\begin{itemize}
    \item Denote $\tau = l_{i_0},l_{i_1},...,l_{i_k}$.
    \item Denote the state transformation of $P_{int}$ (from it's initial to final state - or the one appended to it to be exact) with $T_{int}$.
    \item Let:
    \[
        Option_{int}(S)=S_\tau(\bar{x})\cup
            \{T_{int}(s) \mid s\in S_\tau(\bar{x})\wedge q_{int}(s)\}
    \]
    Inituitively, this transformation expands the set of states possible
    by possibly applying the state transformation of $P_{int}$ to all states.
    \item Base case:
    \[
        T^k_\tau(\bar{x})=\{(\bar{x})\}, R^k_\tau(\bar{x})=true
    \]
    Note how now the transformation function outputs a set rather than a single state.
    This is meant to capture the indetermenism of the interrupt; it represents
    the set of all possible states that can result from an initial (input) state.
    \item Inductive case: Given the functions have been defined for $l_{i_{m+1}}$,
    define them over what is at label $l_{i_m}$:
    \begin{itemize}
        \item \emph{start} or \emph{end}:
        \[
            T^m_\tau(\bar{x})=Option_{int}(T^{m+1}_\tau(\bar{x}))
        \]\[
            R^m_\tau(\bar{x})=R^{m+1}_\tau(\bar{x})
        \]
        \item $\bar{x}:=\bar{y}$:\\
        WLOG we can assume all assignments are to all variables - when this is not the case, we can append identity assignments.
        \[
            T^m_\tau(\bar{x})=Option_{int}(T^{m+1}_\tau(\bar{y}))    
        \]
        \[
            R^m_\tau(\bar{x})=R^{m+1}_\tau(\bar{y})
        \]
        \item $B(\bar{x})$ (boolean branch expression):
        \[
            T^m_\tau(\bar{x})=Option_{int}(T^{m+1}_\tau(\bar{x}))
        \]\[
            R^m_\tau(\bar{x})=
            \begin{cases}
                R^{m+1}_\tau(\bar{x})\wedge[\exists s\in B(T^{m+1}_\tau(\bar{x}))] & \text{if } l_{i_m}\rightarrow^T l_{i_{m+1}}\\
                R^{m+1}_\tau(\bar{x})\wedge[\exists s\in \neg B(T^{m+1}_\tau(\bar{x}))] & \text{if } l_{i_m}\rightarrow^F l_{i_{m+1}}
            \end{cases}
        \]
        Inituitively; we require to get to the conditional label, and then have the
        boolean condition satisfiable by one of the possible states.

        
    \end{itemize}
\end{itemize}


Now we define a modified floyed proof system which follows the steps
on input $(P,P_{int})$:
\begin{enumerate}
    \item Choose a set of cut points s.t.:
    \begin{itemize}
        \item The set contains all initial and final states.
        \item Every cycle in the graph $CFG'(P,P_{int})$ contains at least one cut point.
        \item For every cut point $l$ find an inductive assertion $I_l(\bar{x})$.
            Additionally it is required that: $I_{l_0}(\bar{x}) = q_1(\bar{x}), I_{l_*}(\bar{x}) = q_2(\bar{x})$ where $l_0$
            is the initial state and $l_*$ is a terminal state.
        \item For every simple path between two cut points $l_{i_m}, l_{i_j}$,
        \[
            [I_{l_{i_m}}(\bar{x})\wedge R^m_\tau(\bar{x})\rightarrow I_{l_{i_j}}(\bar{x})]
        \]
    \end{itemize}
\end{enumerate}

The proof system is sound similarly to the original one; 
after the conditions have been shown - we know that for any path from
$l_0$ to $l_*$, $I_{l_0}(\bar{x})\rightarrow I_{l_*}(\bar{x})$ from closure on transitivity
of  the cut points conditions.\\
Since the initial and final conditions are the same as $q_1(\bar{x})$ and $q_1(\bar{x})$ -
we have '$\{q_1(\bar{x})\}(P,P_{int})q_1(\bar{x})$'.

The new proof system is also reasonably complete, as haveing an interrupt which an [\emph{false}]
predicate would mean our proof system is equivalent to the original one.

\section*{Question 3}

\end{document}